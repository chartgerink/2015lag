\section*{Aggregate model results}
Poisson model estimates for all journals together indicate that both review- and production time are only predicted by year. Coefficients in Table \ref{tab:tab2} indicate negligible predictive effects of number of authors, number of pages, and presence of competing interests (i.e., $b \leq |.017|$). Dummy coefficients indicate that review time has increased, whereas production time has fluctuated around 50 days. Besides the effect of year, the results indicate review time is a random process.

\begin{table}
\caption{Table 2. Poisson regression model estimates for review- and production time.}
\label{tab:tab2}
\begin{tabular}{ c c c }
             & Estimate (review) & Estimate (production) \\
    Intercept & 4.18370 & 4.24677 \\
    Authors (centred) & 0.00176 & 0.00582 \\
    Authors$^2$ (centred) & -0.00001 & -0.00001 \\
    Pages (centred) & -0.00084 & 0.00012 \\
    Pages$^2$ (centred) & -0.00010 & -0.00011 \\
    Conflict of interest & -0.01713 & 0.00551 \\
    2004  & 0.68758 & 0.10155 \\
    2005  & 0.74031 & -0.12891 \\
    2006  & 0.69579 & -0.10830 \\
    2007  & 0.55104 & -0.45996 \\
    2008  & 0.62225 & -0.56911 \\
    2009  & 0.59525 & -0.56514 \\
    2010  & 0.66045 & -0.66266 \\
    2011  & 0.65463 & -0.56665 \\
    2012  & 0.73687 & -0.47161 \\
    2013  & 0.73887 & -0.36991 \\
    2014  & 0.77532 & -0.53661 \\
    2015  & 0.84229 & -0.29643 \\
\end{tabular}
\end{table}

The estimated mean review- and production time are depicted in Figure 1. For review time, the estimates are increasing in a non-linear fashion, with a short decreasing trend 2006 and 2008. The estimated mean review time has climbed to approximately 150 days since 2003. Estimated mean production time fluctuates around 50 days. The journal specific model results are described next.
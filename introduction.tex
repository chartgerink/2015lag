Science communication is primarily based on publishing research results in research papers. Anecdotally, authors feel that the publication cycle takes too long \cite{Himmelstein2015-me}. A better understanding of the publication lag could provide solace when feelings of substantial delay occur. Studying the publication cycle could provide an indication whether there are predictive factors of time taken from submission to publication. This paper tries to model such publication times for the Public Libary of Science (PLoS) journals. These journals include PLoS Medicine, PLoS Biology, PLoS ONE, PLoS Pathogens, PLoS Genetics, PLoS Computational Biology, , PLoS Neglected Tropical Diseases, and PLoS Clinical Trials (which was later merged into PLoS Medicine).

The availability of Application Programmatic Interfaces (APIs) to publisher databases has extended the possibilities of quantitative modeling. Previously, a set of articles needed to be sampled or the publisher's website had to be scraped. The information extracted can be unreliable due to sampling; scraping provides the same benefits as APIs but is more technically advanced. PLoS provides users with such an API that contains information to 72 different informational fields for each paper for their full database. These fields include full sections of the paper (e.g., results section), but also the author list, editor, pagecount, etc. In turn, these APIs allow reliable extraction of article metadata, which opens up the avenue of using these metadata for quantitative modeling of publication times. 

Previous research indicated that statistically nonsignificant results take longer to be published \cite{ioannidis1998}, review times have decreased \cite{lyman2013}, and that more figures or tables did not predict publication time \cite{lee2013}. In this paper, I analyze the population of PLoS research articles and split between predicting review time (i.e., time from submission through acceptance) and production time (i.e., time from acceptance through publication). 
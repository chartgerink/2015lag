Science communication is primarily based on publishing research results in research papers. Anecdotally, many authors feel that the publication cycle takes too long; studying the publication cycle could provide an indication whether there are predictive factors of the publication length that are within the authors control. With the rise of the Internet and the availability of Application Programmatic Interfaces (APIs) to publisher databases, quantitative analysis of the publication cycle for all research papers in a journal has become possible. In this study, the publication cycle of research papers in the Public Library of Science (PLoS) is investigated.

Previous research has studied the publication cycle of journals in smaller samples and with focused on length of reviews over years, differences between fields, differences between electronic or pape
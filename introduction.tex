Science communication is primarily based on publishing research results in research papers. Anecdotally, authors feel that the publication cycle takes too long \cite{Himmelstein2015-me}. A better understanding of the publication lag could provide solace when feelings of substantial delay occur, where the main question is whether there are predictive factors of time taken from submission to publication. This paper tries to model publication times for the Public Libary of Science (PLoS) journals. These journals include PLoS Medicine, PLoS Biology, PLoS ONE, PLoS Pathogens, PLoS Genetics, PLoS Computational Biology, , PLoS Neglected Tropical Diseases, and PLoS Clinical Trials (which was later merged into PLoS Medicine).

Previous research indicated that statistically nonsignificant results take longer to be published \cite{ioannidis1998}, review times have decreased \cite{lyman2013}, and that the amount figures or tables does not predict publication time \cite{lee2013}. In this paper, I analyze the population of PLoS research articles and split between predicting review time (i.e., time from submission through acceptance) and production time (i.e., time from acceptance through publication). 
  
\section*{Results}
The collected dataset included information on 135422 research papers. Across all journals, the median publication cycle is 153 days, with the majority of this being the review process (i.e., median 112 days) and not the production process (i.e., median 39 days). Table XX specifies these numbers per journal and indicates PLoS ONE has the fastest review- and production process. PLoS Medicine has the longest review process and exceeds that of PLoS ONE by a median 69 days. PLoS Clinical Trials had the longest production process, exceeding the production process of PLoS ONE by a median 16 days. Note that PLoS Clinical Trials was merged into PLoS Medicine in 2007 and only started in 2006.

These differences in the review- and production speed could be a consequence of increased efficiency or differing selectivity. PLoS ONE contains 117551 papers and is considered a megajournal (i.e., not field specific or selective in topic). PLoS Medicine, for example, contains 1051 papers and is more similar to traditional journals in its criteria for publication (e.g., originality of research; see their guidelines). 
\section*{Results}
\subsection*{Descriptive results}
The collected dataset includes information on 137,468 research papers. Across all journals, the median publication cycle is 152 days, with the majority of this being the review process (i.e., median 111 days) and not the production process (i.e., median 38 days). Table XX specifies these numbers per journal and indicates PLoS ONE has the fastest review- and production process, whereas PLoS Medicine has the longest review process (median difference = 69). PLoS Clinical Trials had the longest production process, compared to PLoS ONE (median difference = 15).

These differences in the review- and production speed could be a consequence of increased efficiency or differing selectivity. PLoS ONE contains 119,435 papers and is considered a megajournal (i.e., not field specific or selective in topic). On the other hand, the other journals are more similar to traditional journals in its criteria for publication (e.g., originality of research). PLoS Medicine, for example, contains 'only' 1,057 papers, indicating a large disparity with PLoS ONE. 

Correlations indicate that the total publication cycle is almost perfectly correlated with the review time ($\rho=.976$). This indicates that $95\%$ of the variance in publication cycle is explained by the review time and that the production process is a random process that adds to publication time.

\subsection*{Model results}
Poisson model estimates across all journals indicate that review times have increased throughout the years and are not predicted by number of authors, number of pages, nor presence of competing interests. Table XX gives the coefficient estimates and indicates that the 
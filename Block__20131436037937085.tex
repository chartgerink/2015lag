\section*{Discussion}
The results of this population level investigation of the PLoS publication cycle indicates that review times have doubled to 150-250 days in the last decade, production time has remained relatively stable at 50 days, and that the publication cycle is not substantially predicted by article metadata. The lack of predictive value of length of a manuscript, number of authors, or the presence of competing interests indicates that the publication cycle might be more a random than a structured process. 

It is noteworthy that, with the development of new editorial systems, the production times for research papers have remained stable in the last decade. Only recently, as of January 1 2015, PLoS has introduced a new set of manuscript guidelines to improve automatization of the production process. Note that the results in this paper show no systematic effect of this, or any previous, adjustment to the production process. The current system might provide this effect in the (near) future, but has not yet.


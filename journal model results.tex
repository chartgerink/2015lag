\section*{Journal model results}
When the results are specified per journal, results model estimates are similar. Most journal specific models included no substantial effect of number of authors, number of pages, or presence of competing interests on either the review- or production time. Only for PLoS Biology the presence of competing interests had a noteworthy effect on production time ($b=.11$). Figure 2 plots the mean estimated review- and production times for each journal.

There is substantial variability in estimated mean review times across journals, but all journals show an increasing time taken to complete the review process. In accordance with the descriptive statistics described earlier, PLoS Medicine is the journal with the longest estimated mean review time, whereas PLoS ONE is the fastest. As of 2015, the review process takes between 150-250 days on average.

The estimated mean production times are highly consistent across journals, but deviates substantially from the aggregate results. Whereas the aggregate results show a curvilinear trend, with an initial decrease in production time and a subsequent inrease, the journal specific results show a stable mean estimate of 50 days. This indicates that 
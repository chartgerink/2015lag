\section*{Method}
Article level data was collected from all PLoS journal research papers with the \texttt{rplos} package \cite[v0.4.7][]{rplos} in \texttt{R} \cite[v3.2.0][]{rcran}. The dataset was collected on June 4, 2015 and is available via \texttt{https://osf.io/53sn9/}. Research papers without a journal name, publication dates, or which had problematic publication dates, were excluded. Problematic publication dates include being published before accepted, accepted before submitted, or accepted at the same time as submitted.

The full publication cycle was split into the review process and the production process. Both these elements of the publication cycle were modeled with Poisson regression models. A Poisson model assumes equal mean and variance (i.e., dispersion $=1$), but the data showed overdispersion (i.e., dispersion $>1$). This was modeled with quasi-likelihood estimation.

Model predictors are year of publication, presence of competing interests, number of pages, and number of authors. The reasoning behind these predictors was as follows. Competing interests could increase publication time when disputed by editors and authors are asked to explain. Number of pages could increase publication time due to longer reviews in both time taken to complete review, the length of the review, and increased production efforts required. Number of authors could influence the time it takes for authors to reach consensus on the response letter and potential other edits during the production process. Squared predictors were included for number of pages and number of authors, due to non-linear relations in the scatterplots with the review- and production times. The number of authors and the number of pages were mean centred to provide meaningful intercept estimates.

Considering that the data are the population of data for PLoS research papers, statistical inference testing is not applied. Additionally, note that PLoS Clinical Trials was merged into PLoS Medicine in 2007 and only started in 2006, which is why other years are not included in estimates for this journal.
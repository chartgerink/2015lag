\section*{Method}
Article level data was collected from all PLoS journals with the \texttt{rplos} package \cite[v0.4.7][]{rplos} in \texttt{R} \cite[v3.2.0][]{rcran}. The dataset was collected on XX June, 2015 and is available via \texttt{https://osf.io/53sn9/} together with annotated data analyses. Research papers without a journal name, or publication dates, or had problematic publication dates were excluded. Problematic publication dates are papers that were published before accepted, accepted before submitted, or accepted at the same time as submitted.

\begin{tabular}{ c c c c c }
          & \# Articles & Publication time & Review time & Production time \\
    ONE   & 119435 & 147   & 106   & 37 \\
    Clinical Trials & 44    & 180.5 & 125   & 52 \\
    Genetics & 4676  & 182   & 131   & 50 \\
    Neglected Tropical Diseases & 2940  & 183   & 133   & 45 \\
    Pathogens & 3926  & 183   & 138.5 & 43 \\
    Biology & 2005  & 190   & 141   & 46 \\
    Computational Biology & 3385  & 199   & 148   & 48 \\
    Medicine & 1057  & 231   & 175   & 47 \\
    Overall & 137468 & 152   & 111   & 38 \\
\end{tabular}

The full publication cycle was split into the review process and the production process for the analyses. Both elements of the publication cycle were modelled with Poisson regression models. Such a Poisson model assumes equal mean and variance (i.e., dispersion $=1$), but the data showed overdispersion (i.e., $>1$). Overdispersion was modelled with quasi-poisson estimation.

Model predictors are year of publication, presence of competing interests, number of pages, and number of authors. The reasoning behind these predictors was as follows. Competing interests could increase publication time when disputed by editors and authors are asked to explain. Number of pages could increase publication time due to longer reviews in both time taken to complete review and the length of the review. Number of authors could influence the time it takes for authors to reach consensus on the response letter. 

Squared predictors were included for number of pages and number of authors. Scatterplot inspection of the predictors with the publication times indicated non-linearity, which is why squared predictors were included. The number of authors and the number of pages were mean centred to provide meaningful intercept estimates.

Considering that the data collected are the population of data for PLoS research papers, statistical inference testing is not applied.

Additionally, note that PLoS Clinical Trials was merged into PLoS Medicine in 2007 and only started in 2006.
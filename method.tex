\section*{Method}
Article level data was collected from all PLoS journal research papers with v0.4.7 of the \texttt{rplos} package \cite{rplos} in \texttt{R} v3.2.0 \cite{rcran}. The dataset was collected on July 1, 2015 and is available via \texttt{https://osf.io/53sn9/}. Research papers without the following were excluded: journal name, publication dates (i.e., submitted, accepted, and published), and problematic publication dates. Problematic publication dates include being published before accepted, accepted before submitted, or accepted at the same time as submitted.

The full publication cycle was split into the review process and the production process. The full publication cycle is the number of days between submission and publication, whereas the review process is the number of days between submission and acceptance; the production process is the number of days between acceptance and publication. The number of days for each element of the publication cycle was modeled with a Poisson regression model. A Poisson regression model is a linear regression model for count variables and assumes equal mean and variance (i.e., dispersion $=1$). The data showed overdispersion (i.e., dispersion $>1$) and quasi-likelihood estimation was used to correct for the violated dispersion assumption.

Model predictors were year of publication, presence of competing interests, number of pages, and number of authors. The reasoning behind these predictors was as follows. Competing interests could increase publication time when disputed by editors and authors are subsequently asked to explain. Number of pages could increase publication time due to longer reviews in both time taken to complete review, the length of the review, and increased production efforts required. Number of authors could influence the time it takes for authors to reach consensus on the response letter and potential other edits during the production process. Squared predictors were included for number of pages and number of authors due to non-linear relations in scatterplots with review- and production days. Additionally, the number of authors and the number of pages were mean centred to provide meaningful intercept estimates.

Considering that the data are the population of data for PLoS research papers, statistical inference testing is not applied. Moreover, note that PLoS Clinical Trials was merged into PLoS Medicine in 2007 and only started in 2006, which is why other years are not included in estimates for this journal.
  
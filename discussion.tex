\section*{Discussion}
The results of this population level investigation of the PLoS publication cycle indicates that review times have doubled to 150-250 days in the last decade, production time has remained relatively stable at 50 days, and that the publication cycle is not substantially predicted by article metadata. The lack of predictive value of length of a manuscript, number of authors, or the presence of competing interests indicates that the publication cycle might be more a random- than a structured process. 

It is noteworthy that, with the development of new editorial systems, the production times for research papers have remained stable in the last decade. Only recently, as of January 1 2015, PLoS has introduced a new set of manuscript guidelines to improve automatization of the production process. Note that the results in this paper show no systematic effect of this, or any previous, adjustment to the production process. The current system might provide this effect in the (near) future, but has not yet.

The increase in review time is substantial and begs the question why this review time has doubled. The increase in review times could be due to any amount of factors, ranging from increased difficulty of finding reviewers through authors taking longer to reply to reviewer comments. That review times are not predicted by the included metadata, however, eliminates these properties of papers as explanatory factors for increased review times. If, for example, the length of the manuscript increased throughout the decade and this explained the increased review time, the effect of year would disappear after controlling for manuscript length. This clearly was not the case.

In sum, authors are left guessing how long it takes for their paper to be published, where this paper indicates that the duration of the publication cycle might be random in some sense. More specifically, publication time seems to only be subject to trends throughout the years and not paper specific characteristics. The trends in the number of review days seem particularly strong, where the doubling of the review time is concerning.
  
  
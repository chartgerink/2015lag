Publications are the driving force in current age academia. However, publishing is a tedious process and can take a considerable amount of time. Previous research has barely investigated whether parts of the publication cycle (i.e., review and production process) can be predicted based on metadata, typically available for all research papers. The predictive value of metadata was investigated in this with three predictors: (i) the number of authors, (ii) the length of the manuscript, and (iii) the presence of competing interests. Additionally, models take into account changes in the publication process throughout the years. Model results indicate that the review and production process are random and cannot be predicted by the included metadata of research papers. Results also indicate review times have doubled throughout the last decade for PLoS journals, which are currently estimated between 150-250 days. Production times, however, have remained highly stable throughout the last decade, around an estimated mean 50 days. The results of these analyses indicate that review- and production times are random, given a certain year-specific mean.
  
\textit{Keywords: publishing, peer-review, plos, metadata.}
  
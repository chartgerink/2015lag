\section*{Abstract}
Publications are the driving force in current age academia. However, publishing is a tedious process and can take a long time. Previous research has barely investigated whether parts of the publication cycle (i.e., review- and production process) can be predicted based on metadata all research papers contain. Metadata investigated in this study are the number of authors, the length of the manuscript, and the presence of competing interests. Additionally, regression models take into account the changes throughout the years. Results indicate that the review- and production process are random processes and cannot be predicted by the metadata of research papers. Review times have doubled throughout the last decade for PLoS journals, which are currently estimated between 15-250 days. Production times, however, have remained highly stable throughout the last decade at an estimated mean 50 days. The results of these analyses indicate that review- and production times are random, given a certain year-specific mean. 
  
\section*{Descriptive results}
The collected dataset includes information on 137,468 research papers. Across all journals, the median publication cycle is 152 days, with the majority of this being the review process (i.e., median 111 days) and not the production process (i.e., median 38 days). Table \ref{tab:tab1} specifies these numbers per journal and indicates PLoS ONE has the fastest review- and production process, whereas PLoS Medicine has the longest review process (median difference = 69). PLoS Clinical Trials had the longest production process, compared to PLoS ONE (median difference = 15).

\begin{table}
\caption{Table 1. Descriptive statistics per journal, with publication-, review-, and production time in median.}
\label{tab:tab1}
\begin{tabular}{ c c c c c }
          & \# Articles & Publication time & Review time & Production time \\
    ONE   & 119,435 & 147   & 106   & 37 \\
    Clinical Trials & 44    & 180.5 & 125   & 52 \\
    Genetics & 4,676  & 182   & 131   & 50 \\
    Neglected Tropical Diseases & 2,940  & 183   & 133   & 45 \\
    Pathogens & 3,926  & 183   & 138.5 & 43 \\
    Biology & 2,005  & 190   & 141   & 46 \\
    Computational Biology & 3,385  & 199   & 148   & 48 \\
    Medicine & 1,057  & 231   & 175   & 47 \\
    Overall & 137,468 & 152   & 111   & 38 \\
\end{tabular}
\end{table}

These differences in the review- and production speed could be a consequence of increased efficiency or differing selectivity. PLoS ONE contains 119,435 papers and is considered a megajournal (i.e., not field specific or selective in topic). On the other hand, the other journals are more similar to traditional journals in its criteria for publication (e.g., originality of research). PLoS Medicine, for example, contains 'only' 1,057 papers, indicating a large disparity with PLoS ONE. 

Correlations indicate that the total publication cycle is almost perfectly correlated with review time ($\rho=.976$). This indicates that $95\%$ of the variance in publication cycle is explained by the review time and that the production process seems an additive random process that is not predicted by the time taken to get a paper accepted.